%%%%%%%%%%%%%%%%%%%%%%%%%%%%%%%%%%%%%%%%%
% Friggeri Resume/CV
% XeLaTeX Template
% Version 1.2 (3/5/15)
%
% This template has been downloaded from:
% http://www.LaTeXTemplates.com
%
% Original author:
% Adrien Friggeri (adrien@friggeri.net)
% https://github.com/afriggeri/CV
%
% License:
% CC BY-NC-SA 3.0 (http://creativecommons.org/licenses/by-nc-sa/3.0/)
%
% Important notes:
% This template needs to be compiled with XeLaTeX and the bibliography, if used,
% needs to be compiled with biber rather than bibtex.
%
%%%%%%%%%%%%%%%%%%%%%%%%%%%%%%%%%%%%%%%%%

\documentclass[]{friggeri-cv} % Add 'print' as an option into the square bracket to remove colors from this template for printing
\usepackage{tikz}

\newcommand{\fc}{\tikz\draw[black,fill=black] (0,0) circle (.5ex);}%
\newcommand{\ec}{\tikz\draw[black,fill=white] (0,0) circle (.5ex);}%

\addbibresource{bibliography.bib} % Specify the bibliography file to include publications

\begin{document}

\header{Michal }{Parusinski}{ingénieur logiciel} % Your name and current job title/field

%----------------------------------------------------------------------------------------
%	SIDEBAR SECTION
%----------------------------------------------------------------------------------------

\begin{aside} % In the aside, each new line forces a line break
\section{contact}
\href{mailto:michal@parusinski.me}{michal@parusinski.me}
\href{https://michal.parusinski.me}{michal.parusinski.me}
\section{langues}
    {\def\arraystretch{0.3}%
    \begin{tabular}{l r}
        \emph{français} & \fc\fc\fc\fc\fc \\
        \emph{anglais} & \fc\fc\fc\fc\fc \\
        \emph{polonais} & \fc\fc\fc\fc\ec \\
        \emph{allemand} & \fc\fc\ec\ec\ec \\
    \end{tabular}%
    }
\section{programmation}
    {\def\arraystretch{0.3}%
    \begin{tabular}{l r}
        \emph{C / C++} & \fc\fc\fc\fc\ec \\
        \emph{Java} & \fc\fc\fc\ec\ec \\
        \emph{Python} & \fc\fc\fc\fc\ec \\
        \emph{Haskell} & \fc\fc\fc\fc\ec \\
        \emph{Javascript} & \fc\fc\fc\ec\ec \\
        \emph{PHP} & \fc\fc\ec\ec\ec 
    \end{tabular}%
    }
\section{informatique}
    {\def\arraystretch{0.3}%
    \begin{tabular}{ r }
        \emph{Unix/Linux} \\
        \emph{Agile} \\
        \emph{Orienté objet} \\
        \emph{Machine Learning} \\
        \emph{Calcul distribué} \\
        \emph{Infra. cloud} \\
    \end{tabular}%
    }
\end{aside}

%----------------------------------------------------------------------------------------
%	EDUCATION SECTION
%----------------------------------------------------------------------------------------

\section{éducation}

\begin{entrylist}

%------------------------------------------------

\entry
{2008--2012}
{master {\normalfont Msci. Mathematics \& Computer Science}}
{Imperial College London}
{Obtenu avec mention très bien}

%------------------------------------------------

\end{entrylist}

%----------------------------------------------------------------------------------------
%	WORK EXPERIENCE SECTION
%----------------------------------------------------------------------------------------

\section{expérience}

\subsection{CDI \& CDD}

\begin{entrylist}

%------------------------------------------------


\entry
{2018--2022}
{ICube-SERTIT}
{Illkirch Graffenstaden, France}
{\emph{Ingénieur logiciel}
\medbreak
Automatisation d'extraction d'information à partir d'image satellites utilisant
    l'apprentissage profond (deep learning) et généralisation de modèle IA en
    utilisant l'augmentation de donnée via GANs.
\smallbreak
    Mise en place d'un SAAS (logiciel en tant que service) pour une chaîne
    bout en bout pour la cartographie de zones brûlées et d'inondations
\smallbreak
Mise en place d'une plateforme de cartes web pour les produits cartographiques
du SERTIT.
}

\entry
{2015--2018}
{Amadeus}
{Sophia-Antipolis, France}
{\emph{Ingénieur logiciel et leadeur technique}
\medbreak
Travail sur les systèmes de réservations de voitures, d'assurances et de 
croisières; à la fois coté backend et frontend.
\smallbreak
Travail de réécriture du système de
    réservation opérant sur mainframe IBM en un système bâti sur une
    architecture moderne distribuée sur des serveurs Linux (TPF Deco).
\bigbreak %
}

\entry
{2013--2015}
{IBM}
{Hursley, Royaume-Uni}
{\emph{Ingénieur logiciel}
\medbreak
Testeur QA pour SPSS Modeler et SPSS Entity Analytics : Création de
    tests et maintien de plusieurs machines de test (Unix and Windows).
\smallbreak
Ingénieur logiciel pour SPSS Modeler : Développement de fonctionnalités comme le
    support pour l'architecture PowerPC et l’intégration avec la plateforme cloud IBM Bluemix.
\bigbreak %
}

\entry
{2012--2013}
{Université Catholique de Louvain}
{Louvain-la-Neuve, Belgique}
{\emph{Assistant de recherche}
\medbreak
Projet de recherche sur les fonctions physiquement
unclonable
(Physically Unclonable Function) :  J’ai utilisé la
\emph{régression logistique} pour attaquer un PUF basé sur la consommation de courant.
}
%------------------------------------------------

\end{entrylist}

\newpage

\subsection{stages}

\begin{entrylist}

\entry
{Été 2011}
{Siemens}
{Princeton, États-Unis}
{\emph{Stagiare d'été}
\medbreak
Contribution à une plateforme d’imagerie médicale pour la simulation du cœur et
l’assistance à la chirurgie cardiaque. Le codage a été effectué
en C++, OpenGL et OpenMP.
\bigbreak
}

\entry
{2011--2012}
{Imperial College London}
{Londres, Royaume-Uni}
{\emph{Assistant au professeur}
\medbreak
    Participation aux cours de 1\textsuperscript{re} année en logique formelle à Imperial College.
\bigbreak
}

\entry
{Été 2010}
{Imperial College London}
{Londres, Royaume-Uni}
    {\emph{Stagiaire Recherche (UROP)}
	\medbreak
    Contribution à une plateforme de simulation de l’océan au sein du groupe
    de recherche AMCG (Applied Modelling \& Computation Group) : Travail 
    sur l'adaptation des données pour la méthode des éléments finis.
	\bigbreak
}

\entry
{Été 2009}
{Personal Audio Ltd.}
{Sydney, Australie}
    {\emph{Ingénieur logiciel}
	\medbreak
    Travail dans une start-up centrée sur l’amélioration de l’effet d'immersion dans les
    jeux vidéos. Création d’une interface utilisateur en QT.
	\bigbreak
}

%------------------------------------------------

\end{entrylist}

%----------------------------------------------------------------------------------------
%	AWARDS SECTION
%----------------------------------------------------------------------------------------

\section{récompenses}

\begin{entrylist}

%------------------------------------------------

\entry
{2008 \& 2011}
{Gloucester Research Prize}
{Imperial College London}
{Distinction pour excellence académique}

%------------------------------------------------

\end{entrylist}


%----------------------------------------------------------------------------------------
%	INTERESTS SECTION
%----------------------------------------------------------------------------------------

\section{intérêts}

\textbf{professionels :} apprentissage automatique (machine learning), sciences
des données, sécurité informatique, serveur informatique, UNIX
\medbreak
\textbf{personels :} arts martiaux, jeux de go, jeux de plateau

%----------------------------------------------------------------------------------------
%	PUBLICATIONS SECTION
%----------------------------------------------------------------------------------------

\section{publications}

\textbf{publication scientifique:} Coauteur de la publication scientifique sur
les fonctions physiquement unclonable
\href{https://perso.uclouvain.be/fstandae/PUBLIS/134.pd}{https://perso.uclouvain.be/fstandae/PUBLIS/134.pd}
\medbreak
\textbf{brevet:} Coauteur du brevet U.S. Patent
\href{http://patft.uspto.gov/netacgi/nph-Parser?Sect2=PTO1&Sect2=HITOFF&p=1&u=/netahtml/PTO/search-bool.html&r=1&f=G&l=50&d=PALL&RefSrch=yes&Query=PN/9582263}{9,582,263} sur la technologie portative

%----------------------------------------------------------------------------------------

\end{document}
