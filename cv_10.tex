%%%%%%%%%%%%%%%%%%%%%%%%%%%%%%%%%%%%%%%%%
% Friggeri Resume/CV
% XeLaTeX Template
% Version 1.2 (3/5/15)
%
% This template has been downloaded from:
% http://www.LaTeXTemplates.com
%
% Original author:
% Adrien Friggeri (adrien@friggeri.net)
% https://github.com/afriggeri/CV
%
% License:
% CC BY-NC-SA 3.0 (http://creativecommons.org/licenses/by-nc-sa/3.0/)
%
% Important notes:
% This template needs to be compiled with XeLaTeX and the bibliography, if used,
% needs to be compiled with biber rather than bibtex.
%
%%%%%%%%%%%%%%%%%%%%%%%%%%%%%%%%%%%%%%%%%

\documentclass[]{friggeri-cv} % Add 'print' as an option into the square bracket to remove colors from this template for printing

\addbibresource{bibliography.bib} % Specify the bibliography file to include publications

\begin{document}

\header{michal}{parusinski}{ingénieur logiciel - technical leader} % Your name and current job title/field

%----------------------------------------------------------------------------------------
%	SIDEBAR SECTION
%----------------------------------------------------------------------------------------

\begin{aside} % In the aside, each new line forces a line break
\section{contact}
Villa Liana,
17 Avenue Paul Arène
Nice, 06000
France
~
+33 (0)6 52 22 90 13
~
\href{mailto:michal@parusinski.me}{michal@parusinski.me}
% \href{http://michal.parusinski.me}{michal.parusinski.me}
\section{langues}
    {\def\arraystretch{0.3}%
    \begin{tabular}{l r}
        \emph{français} & couramment \\
        \emph{anglais} & couramment \\
        \emph{polonais} & couramment \\
        \emph{allemand} & base \\
        \emph{esperanto} & base \\
    \end{tabular}%
    }
\section{programmation}
    {\def\arraystretch{0.3}%
    \begin{tabular}{l r}
        \emph{C / C++} & expert \\
        \emph{Java / Scala  } & avancé \\
        \emph{Python} & avancé \\
        \emph{Haskell} & expert \\
        \emph{PHP} & base \\
    \end{tabular}%
    }
\section{informatique}
    {\def\arraystretch{0.3}%
    \begin{tabular}{ r }
        \emph{Sécurité Informatique} \\
        \emph{Unix/Linux} \\
        \emph{Agile} \\
    \end{tabular}%
    }
\end{aside}

%----------------------------------------------------------------------------------------
%	EDUCATION SECTION
%----------------------------------------------------------------------------------------

\section{éducation}

\begin{entrylist}

%------------------------------------------------

\entry
{2008--2012}
{master {\normalfont Msci. Mathematics \& Computer Science}}
{Imperial College London}
{
    \emph{Obtenue avec mention très bien}
	\medbreak
    1\textsuperscript{re} année : Application de la \emph{méthode des moindres carrées} dans le traitement d’image.
	\smallbreak
    2\textsuperscript{e} année : Application du revêtement universel en topologie dans la théorie des graphes.
	\smallbreak
    3\textsuperscript{e} année : Automatisation des démonstrations et recherche de contre exemples dans la logique de description.
	\smallbreak
    4\textsuperscript{e} année : Factorisation par les courbes elliptiques et la cryptographie RSA.
}

%------------------------------------------------

\end{entrylist}

%----------------------------------------------------------------------------------------
%	WORK EXPERIENCE SECTION
%----------------------------------------------------------------------------------------

\section{éxperience}

\subsection{CDI}

\begin{entrylist}

%------------------------------------------------

\entry
{2015--2017}
{Amadeus}
{Sophia-Antipolis, France}
{\emph{Ingénieur logiciel et leadeur technique}
\medbreak
J’ai travaillé sur les systèmes de réservation de voiture, d'assurance et de croisière; à la
fois côté "backend" et "frontend".
\smallbreak
J’ai travaillé sur le projet de réécrire le système de
    réservation opérant sur mainframe en un système de réservation bâti sur une
    architecture moderne distribuée sur des serveurs Linux (TPF Deco).
\bigbreak %
}

\entry
{2013--2015}
{IBM}
{Hursley, Royaume-Uni}
{\emph{Ingénieur logiciel}
\medbreak
QA pour SPSS Modeler et SPSS Entity Analytics : J’ai participé à la création de
    tests et au maintien de plusieurs machines de test (Unix and Windows).
\smallbreak
Ingénieur logiciel pour SPSS Modeler : Développement de fonctionnalité comme le
    support pour de nouvelles architectures et l’intégration avec le cloud.
\smallbreak
1\textsuperscript{er} projet "Giveback" : Projet avec l’Université de Winchester sur
    l’établissement d’un curriculum de Business Analytics.
\smallbreak
2\textsuperscript{e} projet "Giveback" : Conception d’une plateforme web en PHP, JavaScript et
    Dojo.
\bigbreak %
}

\entry
{2012--2013}
{Université Catholique de Louvain}
{Louvain-la-Neuve, Belgique}
{\emph{Assistant de recherche}
\medbreak
J’ai travaillé sur un projet de recherche sur les fonctions physiquement
unclonable
(Physically Unclonable Function) basé sur la consommation de courant. J’ai utilisé la
\emph{régression logistique} pour attaquer le PUF.
}
%------------------------------------------------

\end{entrylist}

\newpage

\subsection{stages}

\begin{entrylist}

\entry
{Été 2011}
{Siemens}
{Princeton, États-Unis}
{\emph{Stagiare d'été}
\medbreak
J’ai contribué à une plateforme d’imagerie médicale pour la simulation du cœur et
l’assistance à la chirurgie cardiaque. Le codage a été effectué
en C++, OpenGL et OpenMP.
\bigbreak
}

\entry
{2011--2012}
{Imperial College London}
{Londres, Royaume-Uni}
{\emph{Undergraduate Teaching Assistant}
\medbreak
    J’ai assisté avec les cours de 1\textsuperscript{re} année en logique à Imperial College.
\bigbreak
}

\entry
{Été 2010}
{Imperial College London}
{Londres, Royaume-Uni}
    {\emph{Stagiaire Recherche (UROP)}
	\medbreak
    J’ai contribué à une plateforme de simulation de l’océan au sein du groupe
    de recherche AMCG. J’ai travaillé 
    sur l'adaptation des données pour la méthode des éléments finis.
	\bigbreak
}

\entry
{Été 2009}
{Personal Audio Ltd.}
{Sydney, Australie}
    {\emph{Ingénieur logiciel}
	\medbreak
    J’ai travaillé dans une start-up centrée sur l’amélioration de l’effet d'immersion dans les
    jeux vidéos. J’ai créé sur l’interface utilisateur en QT.
	\bigbreak
}

%------------------------------------------------

\end{entrylist}

%----------------------------------------------------------------------------------------
%	AWARDS SECTION
%----------------------------------------------------------------------------------------

\section{récompenses}

\begin{entrylist}

%------------------------------------------------

\entry
{2008,2001}
{Gloucester Research Prize}
{Imperial College London}
{Distinction pour excellence académique}

%------------------------------------------------

\end{entrylist}


%----------------------------------------------------------------------------------------
%	INTERESTS SECTION
%----------------------------------------------------------------------------------------

\section{intérêts}

\textbf{professional:} apprentissage automatique (machine learning), sciences
des données, sécurité informatique, serveur informatique, UNIX\\
\textbf{personal:} arts martiaux, jeux de go, jeux de plateau

%----------------------------------------------------------------------------------------
%	PUBLICATIONS SECTION
%----------------------------------------------------------------------------------------

\section{publications}

\textbf{publication scientifique:} Coauteur de la publication scientifique sur
les fonctions physiquement unclonable
\href{https://perso.uclouvain.be/fstandae/PUBLIS/134.pd}{https://perso.uclouvain.be/fstandae/PUBLIS/134.pd}
\\
\textbf{brevet:} Coauteur du brevet U.S. Patent
\href{http://patft.uspto.gov/netacgi/nph-Parser?Sect2=PTO1&Sect2=HITOFF&p=1&u=/netahtml/PTO/search-bool.html&r=1&f=G&l=50&d=PALL&RefSrch=yes&Query=PN/9582263}{9,582,263} sur la technologie portative

%----------------------------------------------------------------------------------------

\end{document}
